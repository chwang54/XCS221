% This contents of this file will be inserted into the _Solutions version of the
% output tex document.  Here's an example:

% If assignment with subquestion (1.a) requires a written response, you will
% find the following flag within this document: <SCPD_SUBMISSION_TAG>_1a
% In this example, you would insert the LaTeX for your solution to (1.a) between
% the <SCPD_SUBMISSION_TAG>_1a flags.  If you also constrain your answer between the
% START_CODE_HERE and END_CODE_HERE flags, your LaTeX will be styled as a
% solution within the final document.

% Please do not use the '<SCPD_SUBMISSION_TAG>' character anywhere within your code.  As expected,
% that will confuse the regular expressions we use to identify your solution.

\def\assignmentnum{5 }
\def\assignmentname{Course Scheduling}
\def\assignmenttitle{XCS221 Assignment \assignmentnum --- \assignmentname}
\input{macros}
\begin{document}
\pagestyle{myheadings} \markboth{}{\assignmenttitle}

% <SCPD_SUBMISSION_TAG>_entire_submission

This handout includes space for every question that requires a written response.
Please feel free to use it to handwrite your solutions (legibly, please).  If
you choose to typeset your solutions, the |README.md| for this assignment includes
instructions to regenerate this handout with your typeset \LaTeX{} solutions.
\ruleskip

\LARGE
0.a
\normalsize

% <SCPD_SUBMISSION_TAG>_0a
\begin{answer}
  % ### START CODE HERE ###
1. Variables:
    
 Button Variables: `B1, B2, ..., Bm`, each representing a button. The domain of each button variable is `{0, 1}`, where `0` represents not pressing the button and `1` represents pressing the button.
 Light Bulb Variables: `L1, L2, ..., Ln`, each representing a light bulb. The domain of each light bulb variable is `{0, 1}`, where `0` means off and `1` means on. Initially, all light bulb variables are set to `0` (off).

2. Constraints:
    
Each light bulb `Li` has a constraint that depends on its initial state (which is `0` for all bulbs) and the states of the buttons that control it. If `Ti` is the subset of buttons controlling light bulb `i`, then the constraint for light bulb `i` is a function such that `Li` equals the XOR of its initial state with the states of the buttons in `Ti`.
This can be expressed as $Li = \text{initial\_state\_i} \oplus (\text{Bj1} \oplus \text{Bj2} \oplus \ldots \text{Bjx})$ for all $j1, j2, \ldots, jx$ in $Ti$.


3. Goal:
    
The goal of the CSP is to find values for the variables `B1` to `Bm` and `L1` to `Ln` such that all the constraints are satisfied, which in this case means all `Ln` variables are set to `1` (all light bulbs are on).
  % ### END CODE HERE ###
\end{answer}
% <SCPD_SUBMISSION_TAG>_0a
\clearpage

\LARGE
0.b
\normalsize

% <SCPD_SUBMISSION_TAG>_0b
\begin{answer}
  % ### START CODE HERE ###
\section{i}
To determine the number of consistent assignments for the CSP with three variables $X_1, X_2, X_3$ and two XOR constraints $t_1(X) = X_1 \oplus X_2$ and $t_2(X) = X_2 \oplus X_3$, we need to check each possible combination of values for $X_1, X_2,$ and $X_3$ and count those that satisfy both $t_1$ and $t_2$.

The XOR operation yields true (or 1) if and only if the inputs are different. So, for each constraint to be satisfied:
- $t_1(X) = X_1 \oplus X_2$ should be 1, meaning $X_1$ and $X_2$ must be different.
- $t_2(X) = X_2 \oplus X_3$ should be 1, meaning $X_2$ and $X_3$ must be different.

Now, let's enumerate the possibilities:
1. $X_1 = 0, X_2 = 0, X_3 = 0 \Rightarrow t_1 = 0 \oplus 0 = 0$, $t_2 = 0 \oplus 0 = 0$ (Inconsistent)
2. $X_1 = 0, X_2 = 0, X_3 = 1 \Rightarrow t_1 = 0 \oplus 0 = 0$, $t_2 = 0 \oplus 1 = 1$ (Inconsistent)
3. $X_1 = 0, X_2 = 1, X_3 = 0 \Rightarrow t_1 = 0 \oplus 1 = 1$, $t_2 = 1 \oplus 0 = 1$ (Consistent)
4. $X_1 = 0, X_2 = 1, X_3 = 1 \Rightarrow t_1 = 0 \oplus 1 = 1$, $t_2 = 1 \oplus 1 = 0$ (Inconsistent)
5. $X_1 = 1, X_2 = 0, X_3 = 0 \Rightarrow t_1 = 1 \oplus 0 = 1$, $t_2 = 0 \oplus 0 = 0$ (Inconsistent)
6. $X_1 = 1, X_2 = 0, X_3 = 1 \Rightarrow t_1 = 1 \oplus 0 = 1$, $t_2 = 0 \oplus 1 = 1$ (Consistent)
7. $X_1 = 1, X_2 = 1, X_3 = 0 \Rightarrow t_1 = 1 \oplus 1 = 0$, $t_2 = 1 \oplus 0 = 1$ (Inconsistent)
8. $X_1 = 1, X_2 = 1, X_3 = 1 \Rightarrow t_1 = 1 \oplus 1 = 0$, $t_2 = 1 \oplus 1 = 0$ (Inconsistent)

Among these 8 possible assignments, only 2 are consistent (satisfying both constraints). Therefore, there are 2 consistent assignments for this CSP.


\section{ii}

1 initial call

2 calls to assign $X_1$

4 calls to assign $X_3$ (2 for each value of $X_1$)

8 calls to assign $X_2$ (2 for each combination of $X_1$ and $X_3$)

This totals to 15 calls to `backtrack`.

\section{iii}
1 initial call

2 calls to assign $X_1$

2 calls to assign $X_3$ (1 for each value of $X_1$)

2 calls to assign $X_2$ (1 for each combination of $X_1$ and $X_3$)

This totals to 7 calls to `backtrack`.
  % ### END CODE HERE ###
\end{answer}
% <SCPD_SUBMISSION_TAG>_0b
\clearpage

\LARGE
2.d
\normalsize

% <SCPD_SUBMISSION_TAG>_2d
\begin{answer}
  % ### START CODE HERE ###
\# Unit limit per quarter.

minUnits 4

maxUnits 6

\# These are the quarters that I need to fill. It is assumed that

\# the quarters are sorted in chronological order.

register Win2017

register Spr2018

\# Courses I've already taken

taken CS103

taken CS106B

taken CS107

taken CS109

taken CS140

taken CS161

taken CS221

taken MATH51

taken CS145

taken CS124

\# Courses that I'm requesting

request CS224M

request CS224S  

request CS227B   

request CS229A   

request CS232   

request CS244E   

Best schedule 

Here's the best schedule:

Quarter         Units   Course

  Win2017       4       CS229A

  Spr2018       3       CS227B

  Spr2018       3       CS232

 % ### END CODE HERE ###
\end{answer}
% <SCPD_SUBMISSION_TAG>_2d
\clearpage

\LARGE
3.a
\normalsize

% <SCPD_SUBMISSION_TAG>_3a
\begin{answer}
  % ### START CODE HERE ###
In addressing the unsatisfiable constraints in the hospital's residency scheduling, I would advocate for changing factor B, the work to be performed by residents as opposed to other staff. By redistributing some responsibilities to other staff members, the hospital can alleviate the workload on residents, making it more feasible to meet the constraints for rest and working hours. This approach prioritizes the well-being of residents, which is crucial for both their health and the quality of care they provide to patients. It also addresses the root cause of the problem - an excessive workload - rather than merely adjusting the constraints around an unsustainable work schedule.
  % ### END CODE HERE ###
\end{answer}
% <SCPD_SUBMISSION_TAG>_3a
\clearpage

% <SCPD_SUBMISSION_TAG>_entire_submission

\end{document}